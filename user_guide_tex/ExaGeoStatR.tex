	%%%%%%%%%%%%%%%%%%%%%%%%%%%%%%%%%%%%%%%%%
% Arsclassica Article
% LaTeX Template
% Version 1.1 (05/01/2015)
%
% This template has been downloaded from:
% http://www.LaTeXTemplates.com
%
% Original author: Adriano Palombo
%
%
%%%%%%%%%%%%%%%%%%%%%%%%%%%%%%%%%%%%%%%%%

%----------------------------------------------------------------------------------------
%	PACKAGES AND OTHER DOCUMENT CONFIGURATIONS
%----------------------------------------------------------------------------------------

\documentclass[
10pt, % Main document font size
a4paper, % Paper type, use 'letterpaper' for US Letter paper
oneside, % One page layout (no page indentation)
%twoside, % Two page layout (page indentation for binding and different headers)
headinclude,footinclude, % Extra spacing for the header and footer
BCOR5mm, % Binding correction
]{scrartcl}

\usepackage{framed, color}
\usepackage{listings}
\lstset{basicstyle=\ttfamily}
\definecolor{shadecolor}{rgb}{0.59, 0.8, 0.8}

%%%%%%%%%%%%%%%%%%%%%%%%%%%%%%%%%%%%%%%%%
% Arsclassica Article
% Structure Specification File
%
% This file has been downloaded from:
% http://www.LaTeXTemplates.com
%
% Original author:
% Lorenzo Pantieri (http://www.lorenzopantieri.net) with extensive modifications by:
% Vel (vel@latextemplates.com)
%
% License:
% CC BY-NC-SA 3.0 (http://creativecommons.org/licenses/by-nc-sa/3.0/)
%
%%%%%%%%%%%%%%%%%%%%%%%%%%%%%%%%%%%%%%%%%

%----------------------------------------------------------------------------------------
%	REQUIRED PACKAGES
%----------------------------------------------------------------------------------------

\usepackage[
nochapters, % Turn off chapters since this is an article        
beramono, % Use the Bera Mono font for monospaced text (\texttt)
eulermath,% Use the Euler font for mathematics
pdfspacing, % Makes use of pdftex’ letter spacing capabilities via the microtype package
dottedtoc % Dotted lines leading to the page numbers in the table of contents
]{classicthesis} % The layout is based on the Classic Thesis style

\usepackage{arsclassica} % Modifies the Classic Thesis package

\usepackage[T1]{fontenc} % Use 8-bit encoding that has 256 glyphs

\usepackage[utf8]{inputenc} % Required for including letters with accents
%\usepackage[italian]{babel}
\usepackage{graphicx} % Required for including images
\graphicspath{{Figures/}} % Set the default folder for images

\usepackage{enumitem} % Required for manipulating the whitespace between and within lists

\usepackage{lipsum} % Used for inserting dummy 'Lorem ipsum' text into the template

\usepackage{subfig} % Required for creating figures with multiple parts (subfigures)

\usepackage{amsmath,amssymb,amsthm} % For including math equations, theorems, symbols, etc

\usepackage{varioref} % More descriptive referencing

%----------------------------------------------------------------------------------------
%	THEOREM STYLES
%---------------------------------------------------------------------------------------

\theoremstyle{definition} % Define theorem styles here based on the definition style (used for definitions and examples)
\newtheorem{definition}{Definition}

\theoremstyle{plain} % Define theorem styles here based on the plain style (used for theorems, lemmas, propositions)
\newtheorem{theorem}{Theorem}

\theoremstyle{remark} % Define theorem styles here based on the remark style (used for remarks and notes)

%----------------------------------------------------------------------------------------
%	HYPERLINKS
%---------------------------------------------------------------------------------------

\hypersetup{
%draft, % Uncomment to remove all links (useful for printing in black and white)
colorlinks=true, breaklinks=true, bookmarks=true,bookmarksnumbered,
urlcolor=webbrown, linkcolor=RoyalBlue, citecolor=webgreen, % Link colors
pdftitle={}, % PDF title
pdfauthor={\textcopyright}, % PDF Author
pdfsubject={}, % PDF Subject
pdfkeywords={}, % PDF Keywords
pdfcreator={pdfLaTeX}, % PDF Creator
pdfproducer={LaTeX with hyperref and ClassicThesis} % PDF producer
} % Include the structure.tex file which specified the document structure and layout

\hyphenation{Fortran hy-phen-ation} % Specify custom hyphenation points in words with dashes where you would like hyphenation to occur, or alternatively, don't put any dashes in a word to stop hyphenation altogether

%----------------------------------------------------------------------------------------
%	TITLE AND AUTHOR(S)
%----------------------------------------------------------------------------------------

\title{\normalfont\spacedallcaps{ExaGeoStat-R Installation Guide}} 
% The article title

\author{\spacedlowsmallcaps{Version 1.0.0}} % The article author(s) - author affiliations need to be specified in the AUTHOR AFFILIATIONS block

%\date{} % An optional date to appear under the author(s)

%----------------------------------------------------------------------------------------

\begin{document}

%----------------------------------------------------------------------------------------
%	HEADERS
%----------------------------------------------------------------------------------------

\renewcommand{\sectionmark}[1]{\markright{\spacedlowsmallcaps{#1}}} % The header for all pages (oneside) or for even pages (twoside)
%\renewcommand{\subsectionmark}[1]{\markright{\thesubsection~#1}} % Uncomment when using the twoside option - this modifies the header on odd pages
\lehead{\mbox{\llap{\small\thepage\kern1em\color{halfgray} \vline}\color{halfgray}\hspace{0.5em}\rightmark\hfil}} % The header style

\pagestyle{scrheadings} % Enable the headers specified in this block

%----------------------------------------------------------------------------------------
%	TABLE OF CONTENTS & LISTS OF FIGURES AND TABLES
%----------------------------------------------------------------------------------------

\maketitle % Print the title/author/date block

\setcounter{tocdepth}{2} % Set the depth of the table of contents to show sections and subsections only

\tableofcontents % Print the table of contents

%\listoffigures % Print the list of figures

%\listoftables % Print the list of tables

%----------------------------------------------------------------------------------------
%	AUTHOR AFFILIATIONS
%----------------------------------------------------------------------------------------

{\let\thefootnote\relax\footnotetext{* \textit{Extreme Computing Research Center (ECRC), KAUST, Thuwal, Saudi Arabia}}}

%{\let\thefootnote\relax\footnotetext{\textsuperscript{1} \textit{Department of Chemistry, University of Examples, London, United Kingdom}}}

%----------------------------------------------------------------------------------------

\newpage % Start the article content on the second page, remove this if you have a longer abstract that goes onto the second page

%----------------------------------------------------------------------------------------
%	INTRODUCTION
%----------------------------------------------------------------------------------------

\section{Chapter 1. General Information}

\subsection{Introduction}
This document describes how to install ExaGeoStatR on various platforms. One chapter
is dedicated to each operating system.

\subsection{Supported Platforms}
ExaGeoStatR has been tested and is supported on the following operating systems:

\begin{enumerate}
\item MacOS.
\item Linux distributions.
\item Cray XC40 systems.
\end{enumerate}

\begin{shaded}
{\textbf Note:} ExaGeoStat can be run practically on any Linux-like environment with a decent and fairly
up-to-date C++ compiler, for example gcc 5.x. Certain ExaGeoStat-R features (Dense, Diagonal Super Tile(DST), Tile Low Rank(TLR), etc.) depend on the availability of external libraries
(Nlopt, StarPU, Chameleon, HiCMA, Stars-H, etc.)
\end{shaded}

\subsection{Software Dependencies }
\begin{enumerate}
\item {\textbf {Portable Hardware Locality (hwloc):}} a software package provides a portable abstraction of the hierarchical topology of modern architectures.
\item {\textbf {NLopt: a library for nonlinear optimization:}} providing a common interface for a number of different free optimization routines available online as well as original implementations of various other algorithms.
\item {\textbf {GNU Scientific Library (GSL):}} a collection of routines for numerical computing.
\item {{\textbf StarPU:}} a task programming library for hybrid architectures.
\item {{\textbf Chameleon:}} a dense linear algebra software relying on sequential task-based algorithms where sub-tasks of the overall algorithms are submitted to a runtime system.
\item {{\textbf HiCMA:}} Hierarchical Computations on Manycore Architectures library, aims to redesign existing dense linear algebra libraries to exploit the data sparsity of the matrix operator.
\item {{\textbf STARS-H:}} a High performance parallel open-source package of Software for Testing Accuracy, Reliability and Scalability of Hierarchical computations.
\end{enumerate}


\newpage

\section{Chapter 2. macOS}
\subsection{Supported  Releases}
This Chapter provides additional information for installing ExaGeoStat on macOS.
The following release are covered:
\begin{enumerate}
\item macOS high sierra 10.13.
\end{enumerate}

\subsection{Installing the prerequisite Packages}
\begin{enumerate}
\item
\noindent Downloading, Configuring and Building CMake 3.2.3 or higher,
\begin{lstlisting}[language=bash]
$ wget  https://cmake.org/files/v3.12/cmake-3.12.4.tar.gz
$ tar -zxvf cmake-3.12.4.tar.gz
$ cd cmake-3.12.4
$ ./configure --system-curl
$ make -j && sudo make -j install
\end{lstlisting}

\item

\noindent Download an Install Intel MKL library from \url{https://software.intel.com/en-us/mkl} for macOS.

%\noindent . /opt/intel/mkl/bin/mklvars.sh intel64
\noindent \$ export MKLROOT=/opt/intel/mkl

\item
\noindent Downloading, Configuring and Building NLOPT library 2.4.2 or higher,
\begin{lstlisting}[language=bash]
$ wget http://ab-initio.mit.edu/nlopt/nlopt-2.4.2.tar.gz
$ tar -zxvf nlopt-2.4.2.tar.gz
$ cd nlopt-2.4.2
$ ./configure --enable-shared  --without-guile
$ make -j && sudo make -j install
\end{lstlisting}

\item
\noindent Downloading, Configuring and Building GSL library 2.4 or higher,
\begin{lstlisting}[language=bash]
$ wget https://ftp.gnu.org/gnu/gsl/gsl-2.4.tar.gz
$ tar -zxvf gsl-2.4.tar.gz
$ cd gsl-2.4
$ ./configure 
$ make -j && sudo make -j install
\end{lstlisting}

\item
\noindent Downloading, Configuring and Building hwloc 1.11.5 or higher,
\begin{lstlisting}[language=bash]
$ wget https://www.open-mpi.org/software/hwloc/v1.11/
downloads/hwloc-1.11.5.tar.gz
$ tar -zxvf hwloc-1.11.5.tar.gz
$ cd hwloc-1.11.5
$ ./configure
$ make -j && sudo make -j install
\end{lstlisting}

\item
\noindent Downloading, Configuring and Building StarPU 1.2.5 or higher,
\begin{lstlisting}[language=bash]
$ wget http://starpu.gforge.inria.fr/files/starpu-1.2.5/
starpu-1.2.5.tar.gz
$ tar -zxvf starpu-1.2.5.tar.gz
$ cd starpu-1.2.5
$ ./configure  -disable-cuda -disable-mpi --disable-opencl
$ make -j && sudo make -j install
\end{lstlisting}

%
%\begin{shaded}
%{\textbf {StarPU Configuration:}} In the case of GPUs systems, both cuda and opencl should be enabled using -enable option.
%If MPI is required, it should be also enabled using -enable option.
%\end{shaded}


\item
\noindent Download Chameleon, HiCMA, and STARS-H,
\begin{lstlisting}[language=bash]
$ git clone https://github.com/ecrc/exageostat.git
$ cd exageostat
$ git submodule update --init --recursive
\end{lstlisting}

\item
\noindent Configuring and Building Chameleon Software,
\begin{lstlisting}[language=bash]
$ cd exageostat
$ cd hicma
$ cd chameleon
$ mkdir build && cd build
$ cmake ..  -DCHAMELEON_USE_MPI=OFF -DBUILD_SHARED_LIBS=ON 
-DCBLAS_DIR="${MKLROOT}" -DLAPACKE_DIR="${MKLROOT}"
-DBLAS_LIBRARIES="-L${MKLROOT}/lib;-lmkl_intel_lp64;-lmkl_core; 
-lmkl_sequential;-lpthread;-lm;-ldl"
$ make 
$ sudo make install
\end{lstlisting}

%\begin{shaded}
%{\textbf {Chameleon Configuration:}} In the case of GPUs systems,  -DCHAMELEON\_USE\_CUDA=ON
%If MPI is required, it should be also enabled using -DCHAMELEON\_USE\_MPI=ON.
%\end{shaded}

\item
\noindent Configuring and Building STARS-H Library (optional),
\begin{lstlisting}[language=bash]
$ cd exageostat
$ cd stars-h
$ mkdir -p build
$ cd build 
$ cmake .. -DCMAKE_C_FLAGS=-fPIC 
$ sudo make -j && make  install
\end{lstlisting}



\item
\noindent Configuring and Building HiCMA library (optional):
\begin{lstlisting}[language=bash]
$ cd exageostat
$ cd hicma
$ mkdir -p build
$ cd build
$ cmake .. -DBUILD_SHARED_LIBS=ON
$ make 
$ sudo make install
\end{lstlisting}
\end{enumerate}


\begin{shaded}
{\textbf {Note:}} Here, we assume ROOT/SUDO access during installation. if not, 
\begin{enumerate}
\item -DCMAKE\_INSTALL\_PREFIX=\$PWD/installdir should be added to the $cmake$ commands, where $installdir$ is your local installation directory.
\item --prefix=\$PWD/installdir should be added to the $configure$ commands, where $installdir$ is your local installation directory.
\item  $pkgconfig$ for NLOPT, GSL, hwloc, StarPU, Chameleon, STARS-H, HiCMA paths need to be exported to PKG\_CONFIG\_PATH environment variable.
\end{enumerate}
\end{shaded}

\subsection{ExaGeoStat on R}
\begin{enumerate}
\item
\noindent Install latest ExaGeoStat R version hosted on GitHub:
\begin{lstlisting}[language=R]
install.packages("devtools")
library(devtools)
install_git(url="https://github.com/ecrc/exageostatR")
library(exageostat)
\end{lstlisting}


\end{enumerate}


\subsection{Verifying the Installation}
Run one or more examples from Chapter 4.


\section{Chapter 3. Linux}
\subsection{Supported Linux Distributions}
This chapter provides instructions for installing ExaGeoStat on selected Linux distributions:
\begin{enumerate}
\item Ubuntu 16.04 LTS.
\item  Fedora Core 25.
\item Red Hat Enterprise Linux Desktop Workstation 7.x
\item  OpenSUSE 42.

\end{enumerate}

\subsection{Installing the prerequisite Packages}
\begin{enumerate}
\item
\noindent Downloading, Configuring and Building CMake 3.2.3 or higher,
\begin{lstlisting}[language=bash]
$ wget  https://cmake.org/files/v3.10/cmake-3.10.0-rc3.tar.gz
$ tar -zxvf cmake-3.10.0-rc3.tar.gz
$ cd cmake-3.10.0-rc3
$ ./configure 
$ make -j && make -j install
\end{lstlisting}

\item

\noindent Download an Install Intel MKL library from \url{https://software.intel.com/en-us/mkl} for macOS.

%\noindent . /opt/intel/mkl/bin/mklvars.sh intel64
\noindent \$ export MKLROOT=/opt/intel/mkl

\item
\noindent Downloading, Configuring and Building NLOPT library 2.4.2 or higher,
\begin{lstlisting}[language=bash]
$ wget http://ab-initio.mit.edu/nlopt/nlopt-2.4.2.tar.gz
$ tar -zxvf nlopt-2.4.2.tar.gz
$ cd nlopt-2.4.2
$ ./configure --enable-shared  --without-guile
$ make -j && make -j install
\end{lstlisting}

\item
\noindent Downloading, Configuring and Building GSL library 2.4 or higher,
\begin{lstlisting}[language=bash]
$ wget https://ftp.gnu.org/gnu/gsl/gsl-2.4.tar.gz
$ tar -zxvf gsl-2.4.tar.gz
$ cd gsl-2.4
$ ./configure 
$ make -j && make -j install
\end{lstlisting}

\item
\noindent Downloading, Configuring and Building hwloc 1.11.5 or higher,
\begin{lstlisting}[language=bash]
$ wget https://www.open-mpi.org/software/hwloc/v1.11/
downloads/hwloc-1.11.5.tar.gz
$ tar -zxvf hwloc-1.11.5.tar.gz
$./configure
$ make -j && make -j install
\end{lstlisting}

\item
\noindent Downloading, Configuring and Building StarPU 1.2.5 or higher,
\begin{lstlisting}[language=bash]
$ wget http://starpu.gforge.inria.fr/files/starpu-1.2.5/
starpu-1.2.5.tar.gz
$ tar -zxvf starpu-1.2.5.tar.gz
$ cd starpu-1.2.5
$ ./configure  -disable-cuda -disable-mpi --disable-opencl
$ make -j && make -j install
\end{lstlisting}


\begin{shaded}
{\textbf {StarPU Configuration:}} In the case of GPUs systems, both cuda and opencl should be enabled using -enable option.
If MPI is required, it should be also enabled using -enable option.
\end{shaded}


\item
\noindent Download Chameleon, HiCMA, and STARS-H,
\begin{lstlisting}[language=bash]
$ git clone https://github.com/ecrc/exageostat.git
$ cd exageostat-dev
$ git submodule update --init --recursive
\end{lstlisting}

\item
\noindent Configuring and Building Chameleon Software,
\begin{lstlisting}[language=bash]
$ cd exageostat
$ cd hicma
$ cd chameleon
$ mkdir build && cd build
$ cmake ..  -DCHAMELEON_USE_MPI=OFF -DBUILD_SHARED_LIBS=ON 
-DCBLAS_DIR="${MKLROOT}" -DLAPACKE_DIR="${MKLROOT}"
-DBLAS_LIBRARIES="-L${MKLROOT}/lib;-lmkl_intel_lp64;-lmkl_core; 
-lmkl_sequential;-lpthread;-lm;-ldl"
$ make 
$ make install
\end{lstlisting}

\begin{shaded}
{\textbf {Chameleon Configuration:}} In the case of GPUs systems,  -DCHAMELEON\_USE\_CUDA=ON
If MPI is required, it should be also enabled using -DCHAMELEON\_USE\_MPI=ON.
\end{shaded}

\item
\noindent Configuring and Building STARS-H Library (optional),
\begin{lstlisting}[language=bash]
$ cd exageostat
$ cd stars-h
$ mkdir build 
$ cmake .. -DCMAKE_C_FLAGS=-fPIC 
$ make -j && make  install
\end{lstlisting}



\item
\noindent Configuring and Building HiCMA library (optional):
\begin{lstlisting}[language=bash]
$ cd ..
# mkdir build && cd build
$ cmake ..  -DCMAKE_C_FLAGS=-fPIC -DCBLAS_DIR="${MKLROOT}" 
-DLAPACKE_DIR="${MKLROOT}"
-DBLAS_LIBRARIES="-L${MKLROOT}/lib;-lmkl_intel_lp64;-lmkl_core;
-lmkl_sequential;-lpthread;-lm;-ldl"
$ make 
$ make  install
\end{lstlisting}
\end{enumerate}

\begin{shaded}
{\textbf {Note:}} Here, we assume ROOT access during installation. if not, 
\begin{enumerate}
\item -DCMAKE\_INSTALL\_PREFIX=\$PWD/installdir should be added to the $cmake$ commands, where $installdir$ is your local installation directory.
\item --prefix=\$PWD/installdir should be added to the $configure$ commands, where $installdir$ is your local installation directory.
\item  $pkgconfig$ for NLOPT, GSL, hwloc, StarPU, Chameleon, STARS-H, HiCMA paths need to be exported to PKG\_CONFIG\_PATH environment variable.
\end{enumerate}
\end{shaded}


\subsection{ExaGeoStat on R}
\begin{enumerate}
\item
\noindent Install latest ExaGeoStat R version hosted on GitHub:
\begin{lstlisting}[language=R]
install.packages("devtools")
library(devtools)
install_git(url="https://github.com/ecrc/exageostatR")
library(exageostat)
\end{lstlisting}


\end{enumerate}


\subsection{Verifying the Installation}
Run one or more examples from Chapter 4.


%%%%%%%%%%%%%%%%%%%%%%%%%%%%%%%%%%%%%%%%%%%%%%%%%%%
\newpage
\section{Chapter 4. Examples}
\subsection{Example 1: Synthetic generation of Geo-spatial data with MLE exact computation,}
\begin{lstlisting}[language=R]
library("exageostat") 		#Load ExaGeoStat-R lib.
seed         = 0 		#Initial seed to generate XY locs.
theta1       = 1 		#Initial variance.
theta2       = 0.1 		#Initial range.
theta3       = 0.5 		#Initial smoothness.
dmetric      = 0 		#0 --> Euclidean distance, 1--> great circle distance.
n            = 1600 		#n*n locations grid.
ncores       = 2 		#Number of underlying CPUs.
gpus         = 0 		#Number of underlying GPUs.
ts           = 320 		#Tile_size:  changing it can improve the performance. 
p_grid       = 1 		#More than 1 in the case of distributed systems.
q_grid       = 1 		#More than 1 in the case of distributed systems.
clb          = vector(mode="double",length = 3) #Optimization  lower bounds values.
cub          = vector(mode="double",length = 3) #Optimization  upper bounds values.
theta_out    = vector(mode="double",length = 3) #Parameter vector output.
globalveclen = 3*n
vecs_out     = vector(mode="double",length = globalveclen)#Z measurements of n locations.
clb          = as.double(c("0.01", "0.01", "0.01")) 	  #Optimization lower bounds.
cub          = as.double(c("5.00", "5.00", "5.00"))  	  #Optimization upper bounds.
vecs_out[1:globalveclen]     = -1.99
theta_out[1:3]               = -1.99
exageostat_initR(ncores, gpus, ts) 			   #Initiate exageostat instance.
vecs_out     = exageostat_egenzR(n, ncores, gpus, ts, p_grid, q_grid,
theta1, theta2, theta3, dmetric, seed, globalveclen) 	  #Generate Z observation vector.
theta_out    = exageostat_emleR(n, ncores, gpus, ts, p_grid, q_grid,  vecs_out[1:n],  vecs_out[n+1:(2*n)],
vecs_out[(2*n+1):(3*n)], clb, cub, dmetric, 0.0001, 20)    #Exact Estimation (MLE).
exageostat_finalizeR() 					   #Finalize exageostat instance
\end{lstlisting}

========================================================================

\subsection{Example 2: Synthetic generation of Geo-spatial data with MLE TLR computation,}
\begin{lstlisting}[language=R]
library("exageostat") 		#Load ExaGeoStat-R lib.
seed         = 0 		#Initial seed to generate XY locs.
theta1       = 1 		#Initial variance.
theta2       = 0.03 		#Initial range.
theta3       = 0.5 		#Initial smoothness.
dmetric      = 0 		#0 --> Euclidean distance, 1--> great circle distance.
n            = 900 		#n*n locations grid.
ncores       = 4 		#Number of underlying CPUs.
gpus         = 0 		#Number of underlying GPUs.
ts           = 320 		#Tile_size:  changing it can improve the performance. 
lts          = 600 		#TLR_Tile_size:  changing it can improve the performance. 
tlr_acc      = 7 		#approximation accuracy 10^-(acc).
tlr_maxrank  = 450 		#Max rank.
p_grid       = 1 		#More than 1 in the case of distributed systems.
q_grid       = 1 		#More than 1 in the case of distributed systems.
clb          = vector(mode="double",length = 3) #Optimization  lower bounds values.
cub          = vector(mode="double",length = 3) #Optimization  upper bounds values.
theta_out    = vector(mode="double",length = 3) #Parameter vector output.
globalveclen = 3*n
vecs_out     = vector(mode="double",length = globalveclen)#Z measurements of n locations.
clb          = as.double(c("0.01", "0.01", "0.01")) 	  #Optimization lower bounds.
cub          = as.double(c("5.00", "5.00", "5.00"))  	  #Optimization upper bounds.
vecs_out[1:globalveclen]     = -1.99
theta_out[1:3]               = -1.99
exageostat_initR(ncores, gpus, ts) 			  #Initiate exageostat instance.
vecs_out     = exageostat_egenzR(n, ncores, gpus, ts, p_grid, q_grid,
theta1, theta2, theta3, dmetric, seed, globalveclen) 	  #Generate Z observation vector.
theta_out       = exageostat_tlrmleR(n, ncores, gpus, lts, p_grid, q_grid,  vecs_out[1:n],  vecs_out[n+1:(2*n)],  vecs_out[(2*n+1):(3*n)],
 clb, cub, tlr_acc, tlr_maxrank,  dmetric, 0.0001, 20)    #TLR Estimation (MLE).
exageostat_finalizeR() 					  #Finalize exageostat instance.
\end{lstlisting}



========================================================================


\subsection{Example 3: Synthetic generation of Geo-spatial data with MLE DST computation,}
\begin{lstlisting}[language=R]
library("exageostat") 		#Load ExaGeoStat-R lib.
seed         = 0 		#Initial seed to generate XY locs.
theta1       = 1 		#Initial variance.
theta2       = 0.1 		#Initial range.
theta3       = 0.5 		#Initial smoothness.
dmetric      = 0 		#0 --> Euclidean distance, 1--> great circle distance.
n            = 1600 		#n*n locations grid.
ncores       = 2 		#Number of underlying CPUs.
gpus         = 0 		#Number of underlying GPUs.
ts           = 320 		#Tile_size:  changing it can improve the performance. 
p_grid       = 1 		#More than 1 in the case of distributed systems.
q_grid       = 1 		#More than 1 in the case of distributed systems.
clb          = vector(mode="double",length = 3) #Optimization  lower bounds values.
cub          = vector(mode="double",length = 3) #Optimization  upper bounds values.
theta_out    = vector(mode="double",length = 3) #Parameter vector output.
globalveclen = 3*n
vecs_out     = vector(mode="double",length = globalveclen)#Z measurements of n locations.
clb          = as.double(c("0.01", "0.01", "0.01")) 	  #Optimization lower bounds.
cub          = as.double(c("5.00", "5.00", "5.00"))  	  #Optimization upper bounds.
vecs_out[1:globalveclen]     = -1.99
theta_out[1:3]               = -1.99
exageostat_initR(ncores, gpus, ts) 			   #Initiate exageostat instance.
vecs_out     = exageostat_egenzR(n, ncores, gpus, ts, p_grid, q_grid,
theta1, theta2, theta3, dmetric, seed, globalveclen) 	  #Generate Z observation vector.
theta_out    = exageostat_dstmleR(n, ncores, gpus, ts, p_grid, q_grid,  vecs_out[1:n],  vecs_out[n+1:(2*n)],
vecs_out[(2*n+1):(3*n)], clb, cub, dmetric, 0.0001, 20)    #DST Estimation (MLE).
exageostat_finalizeR() 					   #Finalize exageostat instance
\end{lstlisting}
========================================================================


\subsection{Example 4: Synthetic generation of measurements based on given Geo-spatial data locations with MLE Exact computation,}
\begin{lstlisting}[language=R]
library("exageostat") 		#Load ExaGeoStat-R lib.
seed         = 0 		#Initial seed to generate XY locs.
theta1       = 1 		#Initial variance.
theta2       = 0.1 		#Initial range.
theta3       = 0.5 		#Initial smoothness.
dmetric      = 0 		#0 --> Euclidean distance, 1--> great circle distance.
n            = 1600 		#n*n locations grid.
ncores       = 2 		#Number of underlying CPUs.
gpus         = 0 		#Number of underlying GPUs.
ts           = 320 		#Tile_size:  changing it can improve the performance. 
p_grid       = 1 		#More than 1 in the case of distributed systems.
q_grid       = 1 		#More than 1 in the case of distributed systems.
clb          = vector(mode="double",length = 3) #Optimization  lower bounds values.
cub          = vector(mode="double",length = 3) #Optimization  upper bounds values.
theta_out    = vector(mode="double",length = 3) #Parameter vector output.
globalveclen = 3*n
x               = rnorm(n = globalveclen, mean = 39.74, sd = 25.09)     #X dimension vector of n locations.
y               = rnorm(n = globalveclen, mean = 80.45, sd = 100.19)    #Y dimension vector of n locations.
vecs_out     = vector(mode="double",length = globalveclen)#Z measurements of n locations.
clb          = as.double(c("0.01", "0.01", "0.01")) 	  #Optimization lower bounds.
cub          = as.double(c("5.00", "5.00", "5.00"))  	  #Optimization upper bounds.
vecs_out[1:globalveclen]     = -1.99
theta_out[1:3]               = -1.99
exageostat_initR(ncores, gpus, ts) 			   #Initiate exageostat instance.
vecs_out        = exageostat_egenz_glR(n, ncores, gpus, ts, p_grid, q_grid,
x, y, theta1, theta2, theta3, 
dmetric, globalveclen) 		  #Generate Z observation vector based on given locations.
theta_out    = exageostat_emleR(n, ncores, gpus, ts, p_grid, q_grid,  vecs_out[1:n],  vecs_out[n+1:(2*n)],
vecs_out[(2*n+1):(3*n)], clb, cub, dmetric, 0.0001, 20)    #Exact Estimation (MLE).
exageostat_finalizeR() 					   #Finalize exageostat instance
\end{lstlisting}

%----------------------------------------------------------------------------------------
%	BIBLIOGRAPHY
%----------------------------------------------------------------------------------------

%\renewcommand{\refname}{\spacedlowsmallcaps{References}} % For modifying the bibliography heading

%\bibliographystyle{unsrt}

%\bibliography{sample.bib} % The file containing the bibliography

%----------------------------------------------------------------------------------------

\end{document}
